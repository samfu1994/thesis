%# -*- coding: utf-8-unix -*-
%%==================================================
%% chapter04.tex for SJTU Master Thesis
%%==================================================

%\bibliographystyle{sjtu2}%[此处用于每章都生产参考文献]
\chapter{设计情绪实验}
\label{chap:chap4}

\section{收集情绪刺激视频}
	由于需要采集人的情绪,所以进行实验的第一步是收集视频作为诱发被试情绪的材料。实验室所选择实验材料的标准如下:
	\begin{enumerate}
	\item 为了避免由于对英语熟悉程度不同而造成的理解差异,我们仅选择汉语配音的视频。
	\item 由于本实验被试几乎全部是在校本科生和研究生,所以实验材料尽量选取符合大学生价值取向,容易被大学生理解并激发其相应情绪的。
	
	\item 影片符合基本当代价值取向,不包含极端内容和思想。
	\item 每个片段长度尽量选取在1-4分钟,既可以采取到相应情绪的数据,也保证不会有很多冗余内容造成了中性(无聊)的情绪。
	\end{enumerate}
	
\section{情绪刺激视频筛选}
	由于采取到的负荷上述要求的视频片段的数量非常多,如果采用全部片段作为实验素材,那么实验的耗时会太长。因此,需要筛选出最合适的素材。为了解决这个问题,实验室采用Alexander S.所提出的标准来评价收集到的素材,并根据来自志愿者的评价来对这些素材进行评估。
	志愿者需要填写的问卷表对每个影片素材都有二十多项标准,包含了譬如激发程度,感兴趣程度,热情程度,哦那个剧程度,失落程度,警惕程度等等。每一项都从1-5中选一数字进行打分,数字越大则代表感受到的响应程度越深。最后,我们对这些问卷就行统计和评估,选出了最适合的十五段情绪。去除了容易令人迷惑的片段,最终结果和目的则是增加了实验数据的可信度。
	
\section{情绪刺激实验流程}
	本论文中采用ESI Neuroscan平台进行EEG信号的采集和记录。在被试戴好脑电帽,所有电极都测试妥当之后,他会进入实验室的真车环境中观看大屏幕。每个被试进行三次实验,每次实验中均为15个视频片段。除此之外,每个片段开头会有剧情简介,让被试有心理准备,获得更好的情绪激发效果。
\section{本章小结}
	本章主要介绍了实验素材的收集,筛选和实验的基本流程。