%# -*- coding: utf-8-unix -*-
%%==================================================
%% chapter06.tex for SJTU Master Thesis
%%==================================================

%\bibliographystyle{sjtu2}%[此处用于每章都生产参考文献]
\chapter{脑电与情绪相关背景}
\label{chap:chap2x}

\section{大脑结构和对应功能分布}
	脑是人类高级神经系统的主要组成部分,它包括大脑,小脑和脑干。人类大脑主要分成了脑缘系统,大脑皮质,和脑核。
	
	大脑皮质在大脑的表层上。大脑的中间有一道纵向裂痕,将大脑分成左右两个半球,也就是我们所说的大脑半球。大脑皮质肩负着人类的高级情感和神经方面的功能。大脑皮质主要分为左右半球两部分,每个半球又分别包含了额叶,顶叶,枕叶和颞叶。
	
	在相关学术领域,这四个脑区的研究已经较为深入,针对着四个脑区的功能,目前被广泛接受的理论如下:
	\begin{itemize}
		\item 额叶。
		
			负责高级认知功能,譬如学习,语言,决策,运动,情绪等。
		\item 顶叶。
		
			负责躯体感觉功能,视觉功能,空间位置处理功能等。
		\item 颞叶。
			
			负责听觉,嗅觉等功能。另外还负责了长期记忆功能等。
		\item  枕叶。
		
			高级视觉处理的功能。
	\end{itemize}
	
\section{脑电产生原理}
	
	脑电是头皮部位的电位活动。测量脑电,测量的是大脑神经元里面产生的电流和典雅波动。
	
	大脑里的电荷分布在了数亿的神经元中,而神经元之间通过膜转运蛋白对离子的转运来放点。神经元无时无刻不在与外环境进行离子交换,来维持静息电位或者激发动作电位。根据电荷同性相斥异性相吸的原理,当带有相同电荷的离子离开神经元的时候,就会排斥周围环境相同电荷的离子,通过这样一个链不断传递下去,就形成了波,而上述过程就是容积导电。当离子形成的波传播到了头皮和我们的采集电极,电极里的电子就会运动。而这种电子运动会让不同位置上的电极产生电压差,我们可以测量出来这个电压差,并且随着时间记录,这就是脑电图。
	
	一个神经元产生的电压太小,无法被脑电图记录下来,所以我们测量出来的脑电图,往往是非常多空间中邻近的神经元的同步活动。如果神经元不是空间中邻近,那么他们的同步活动很难被记录。皮质椎体神经元排列整齐,并且行为往往一致。所以经常是脑电信号产生的主要神经元。 由于电压随着距离的增加而平方衰减,大脑内部的脑电相对于靠近头骨的脑电更难被记录。
	
\section{采集脑电}

	常用的脑电采集的方法是用胶状脑电膏把电极和头皮粘在一起。这样一来因为头皮到电极的电阻被减小,脑电信号会更容易采集。这种方法虽然能采集到清晰的脑电信号,但是由于脑电膏涂抹的工作十分复杂,并且被试的实验过程将会变得不舒适,现在实验室采用的是干电极直接与头皮接触采集脑电信号,避免了脑电膏的使用。
	
	我们根据国际上被广泛使用的10-20系统来规范电极在头皮上的摆放位置[参考文献,drn32]。这个规范对电极位置的安排十分对称,使得采集到的脑电信号较为完善。 根据这个规范,我们选用了62导电极的采集方法。下图所示是62导电极采集方法的图示。
	
\section{脑电信号频域特征}
	
		不同频带上的脑电信号有不同的生理意义,通常我们将1-50Hz的脑电信号分成了五个频段,分别是Delta, Theta, Alpha, Beta, Gamma。 特征如下所示:
		\begin{itemize}
			\item \textbf{Delta频带}
			
				1-4Hz。Delta频带通常出现在成年人的前额或者婴儿的后脑的位置,并且波幅较大。该频带的脑电往往在成年人睡眠时出现,也可能出现在成年人连续注意的时候。
				
			\item \textbf{Theta频带}
				
				4-8Hz。Theta频带通常出现在幼儿,困倦和刚刚醒来的儿童和成年人中,一般情况下,出现这个频带的脑电信号时,人的意识比较模糊,反应比较迟钝。
				
			\item \textbf{Alpha频带}
			
				8-14Hz。Alpha频带通常被看作是脑电图的基本频带,出现在头的后部或者两边,中间部位较少出现。该频段出现时,对于该频段脑电波在左右脑分布的情况,结果是非主导侧的波幅较大。出现了这个频带的脑电信号时,人们往往在进行休息,反射较为简单。
				
			\item \textbf{Beta频带}
			
				14-31Hz。Beta频带的波幅较小,通常出现在大脑两侧的位置,并且分布对称,前侧的波幅最强。 出现该频带的脑电信号时,人通常处于注意力高度集中的状态,譬如警惕,思考,忙碌等。
				
			\item \textbf{Gamma频带}
			
				31-50Hz。该频带的脑电信号往往出现在负责躯体感觉的皮层,并且往往在人进行多个模态信息的处理中(如嗅觉和听觉,听觉和视觉)出现,同时也常在短时记忆发生的时候出现。
		\end{itemize}
\section{本章小结}

	本章介绍了本文所需要的脑电背景知识,包括在脑电的研究背景。主要介绍了大脑的结构和分区等生理学知识,以及脑电信号采集的原理,脑电产生的生理学远离,并着重介绍了脑电的频域特性。不同区域的差异和不同频段的差异同样是情绪识别研究的热点,而脑电的产生原理和脑电的采集更是实验室的的数据来源,与本文的数据密切相关。