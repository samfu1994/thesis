%# -*- coding: utf-8-unix -*-
%%==================================================
%% chapter02.tex for SJTU Master Thesis
%%==================================================

%\bibliographystyle{sjtu2}%[此处用于每章都生产参考文献]
\chapter{情绪及情绪刺激}
\label{chap:chap2}
	


\section{情绪定义}
情绪在心理学上的定义是:人们对需求和客观事物两者关系的应激性反应,是一种每个人不同的主观感受、也是每个人都有的生理反应、还是认知的互动,并且有表达出特定的行为的趋势。同时,也有很多相关学者认为情绪和认知联系紧密,不可分割。另外,情绪表达与情绪不同,前者是一个人的内在情绪通过表情、动作、语言等方式表现出来。在意识和价值观的作用下,每个人所表达的情绪可能是强化或者弱化的真正的情绪。我们可以总结出以下几点:
\begin{itemize}
  \item 情绪是本身对外界的一种自然反应。情绪没有好坏对错,而是一种离散的对外界或者内部的刺激而自然产生的反应。
  \item 情绪是外来刺激和内在认知的一种互动。正面或负面情绪的出现,分别是自身对需求得到满足和未得到满足时产生的生理反应。这也就是情绪的生理特征。
  \item 情绪有情绪表达的趋势。情绪是由外而内的感受和刺激带来的,然后又由内而外的表达。这也就是情绪的表达和动作趋势。
\end{itemize}

\section{情绪分类与情绪分类模型}
	一直以来,都有两种不同的观点来进行情绪的度量。一种认为情绪是离散并且是可以相互叠加从而派生出新的情绪的。另一种是认为情绪是连续的并且可以进行测量。
	 \subsection{基本情绪}
	 P. Ekman提出的理论认为情绪是离散可叠加并且可以独立的被测量。他最有影响力的研究成果是有六类基本情绪是跨越了种族和国界而广泛适用于人类中的,基于他的研究成果,他提出论点认为人的情绪需要分成六种:愤怒,悲伤,高兴,恐惧,厌恶和惊愕。
	 \subsection{多维度情绪表示}
	 利用数据可视化的原理,我们可以在二维模型上将所有情绪描述出来。自然地,相似的情绪状态在二维坐标系中应该距离更近,而相差愈大的情绪状态在坐标系中应该距离更远。通常,心理学家们把这两个用于可视化情绪的维度分别叫做警觉度和效价。后者主要表示情绪的消极/积极程度,而前者代表情绪给人带来刺激的强度。可以用下图(图2-1)表示。
	 \subsection{情绪激发}
	 虽然在每天平淡的生活中,每个人无时无刻都不被感情充斥着,但是感情却很难完全由自己控制。因此,在实验搜集数据时,我们对能够快速调动并激发实验参与者某些特定情绪和感情的方法有迫切的需求,并且要有较好的可依赖性,鲁棒性和可持续性。在观察了现有研究者尝试过的多种方法以后,论文决定采用视频库作为被试者观看的素材,也就是刺激材料。
	 通过广泛的搜索,本文作者找到了大量汉语作为音频的国内外电影或微电影,排除对英语接受程度不同这个额外变量的影响。不同于P. Ekman提出的六种基本情绪的分类,本文选取高兴,悲伤,中性三种区别最为显著的基本兴趣。在数据收集阶段,实验室通过询问大量与实验参与者各方面条件都类似的学生的评价,根据评价的高低最终选出了15段视频片段作为最终的实验素材。
	 最后,为了保证数据的可信程度和获取足够数量的数据,每个被试都分别进行三次的实验。这也为我们下一步继续研究基于个体的情绪识别研究提供了数据。
\section{本章小结}
	本章对情绪的定义,种类,分类的不同方式和情绪的二维可视化方法做了基本介绍。基于上述的情绪评价模型,本文针对相关需求使用了特定的素材,最终采集到了适合进行EEG和眼动仪信号多模态分析的数据。

