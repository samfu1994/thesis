%# -*- coding: utf-8-unix -*-
%%==================================================
%% chapter06.tex for SJTU Master Thesis
%%==================================================

%\bibliographystyle{sjtu2}%[此处用于每章都生产参考文献]
\chapter{结论与展望}
\label{chap:chap8}
\section{课题结论}{
	本课题主要对脑电信号的的情绪识别进行了研究,为了进一步提高现有的情绪识别结果,我们将多模态深度学习的相关算法应用于该课题上。在前人提出的多模态深度学习算法的基础上,我们又尝试了不同的激励函数和不同深度的网络,最后发现ReLU的结果好于sigmoid,不同深度的特征也有不同的表现,
	
	最终,实验和测试得到的结果比较符合理论的分析和我们的预期,并且相比基准方法和之前的方法有所提高。并且多次实验后,最后选用的库的训练速度较快,训练时间不会很长,令人满意。
}
\section{未来工作展望}

	本课题所提出的算法和各个优化使得最终结果有了提高,然而距离多模态深度学习能够达到的极限还有很大距离。所以,我们根据当前的结果提出了以下有待提高的地方,希望之后的工作中可以参考以下改进意见,继续改进这个算法的性能。
	\begin{enumerate}
		\item 因为前期工作准备时间较长,尝试了多个库,所以导致后期时间较紧,来不及进行调参,包括深度自编码器的参数和分类器的参数,譬如隐层的数目,学习速率,分类器核函数等等。下面如果进行详尽的调参,相信结果一定会有进一步的提高。
		\item 一直以来实验室对结果的比较都是把五个频段的结果分别列出,再列出全频段的结果进行对比。由于实验时间不足,我们只对部分频段的结果进行了实验和分析。未来可以对五个频段进行分别对比,寻找对于这个分类算法的最佳频段。
		\item 现在算法的结果是多模态经过深度网络后融合的分类结果,我们实验结果是直接与不经过任何网络的特征进行对比,这样一来,改变的环境太多。下面如果要通过对比显示多模态于单模态的意义,我们应该对单纯的脑电和眼电分别加上这个算法,用深度自编码器对单模态特征进行特征提取,再对比结果,这样一来改变的只是由单模态到多模态。 而直接把多模态特征拼接,对比我们的多模态深度算法得到的结果,显示的是深度自编码器网络的意义。通过这样的控制变量的方法,我们能进一步说明多模态的意义。
		\item 为了显示我们的深度自编码器学习到的特征更有意义,未来可以写聚类算法分别对原始特征和我们学习到的特征进行对比,自然的,我们学习到的特征会更明显的分为三个类别。通过这样的数据可视化方法可以说明我们学习到了更优秀的特征。
		\item 提取后的特征现在只是用SVM来进行分类,未来可以考虑尝试多个分类器,譬如多层感知机等。
		\item 未来可以进一步实现寻找共同特征表达等任务,即测试和训练所用的数据来自不同模态。
	\end{enumerate}