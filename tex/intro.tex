%# -*- coding: utf-8-unix -*-
%%==================================================
%% chapter01.tex for SJTU Master Thesis
%%==================================================

%\bibliographystyle{sjtu2}%[此处用于每章都生产参考文献]
\chapter{绪论}
\label{chap:intro}
	一直以来,情绪都被认为是高级生命体独有的特征。而它是极其复杂的心理现象,是一种多成分、多维度、多种类组合的心理过程。每个人的情绪不但与自己的认知息息相关,每种情绪还可以相互组合从而派生出更复杂而高级的情绪。心理现象是脑的功能,所以大脑是产生不同情绪的根源所在,所以基于脑电信号的情绪识别方法无疑最可能成为有最高准确率的方法之一。因此,建立一套准确,方便同时广泛适用的模型对于情绪的研究来说意义重大。此时,基于脑电信号分析的算法可以发挥效用。


\section{研究背景与意义}

	每个人无时无刻都不在产生情绪,情绪在日常生活中时刻都在陪伴着我们,它不但影响着我们的一举一动,还会对我们的健康产生积极或者消极的影响。积极的情绪可以带给人民愉悦感,获得好的心情并且提高工作效率。并且有科学证据表明,愉快喜悦等积极情绪还可以使得伤口加快愈合,促进疾病痊愈。与此相反,消极的情绪会阻碍人们的个性发展,降低人们的自信和自我评价,还会影响人们的认知思维水平,妨碍日常生活。长期生活在抑郁、忧郁或恐惧下的人或性格古怪,社交能力较差,从而影响他们的生活质量。
	情绪可以有多种方式被表达,譬如语言文字,肢体语言,面部表情等等。但是与脑电图(Electroencephalograph, 简称:EEG)信号相比,提供的数据有特征不突出,没有明显的模式的特点,加大了我们使用这些数据的难度,所以利用EEG信号来识别情绪可以获得比较好的准确率。而这项研究也成为当今EEG领域的热点之一。
	EEG 信号是由电极采集在人体脑部自身产生的微弱生物电,然后在头皮处经放大后而得到数据。 这种信号最初是用于癫痫、脑血管疾病的检查,而现在由于它的时效性和前文所述的多种优点, 已经在多个领域被广泛利用,包括情绪识别,医疗,虚拟现实等等。而利用脑电信号来进行情绪识别的最大优点就是可以在保持识别过程的迅速的基础上,有更好的准确率,为进一步的各个细分领域的研究打下了坚实的基础。举例来说,譬如帮助抑郁症患者或者受到重大精神打击的人进行情绪分析,从而让医生了解病人真正的情绪,提供更好的治疗方案。或者对正在执行任务的警察保持情绪的观察,如果出现不正常的愤怒等情绪可以利用合适的方法来提醒。或者在远程教育方面,实时地情绪识别可以帮助老师快速地了解学生们的情绪状态,并以此为依据来在课上调整授课内容与方式, 保证教学的质量和灵活性。
	与此同时,随着机器的计算能力的提高,深度学习的方法例如深度置信网络,深度自编码器等受到研究人员的青睐。但是如果只是用单一模态的输入进行学习,由于数据的分布总是相似的,所以提供的信息和学习能力都是很有限的。相反,不同模态譬如声音,文字,图像,脑电,眼动轨迹带有不同信息,所以提供的数据分布各不相同,他们各自提供的信息不但有交集,还互相补充。因此,如果能够利用多模态输入信号的融合,我们不仅可以提取出多模态共有的特征表达,从而找出它们代表的现实世界中的意义,还可以利用多个模态不同信号的互补信息来提高情绪识别任务的准确率。
	所以,本课题的目标在于把眼动轨迹和EEG信号进行融合,找到合适且快速的融合方法,从而利用多模态的信息来进行情绪识别,这样就可以利用眼动轨迹和EEG信号所能提供的互补信息来获得比只是利用单一模态信息更好的表现。
	
\section{国内外研究现状}

	Hinton[1]在2006年发表的文章,首次提出了逐层的无监督训练深度新年网络的方法,是近些年深度学习越来越受到研究者关注的开端。
	
	在图像处理领域,Alex K.[2 http://papers.nips.cc/paper/4824-imagenet-classification-with-deep-convolutional-neural-networks]于2012年在NIPS发表的文章训练了深度卷积神经网络,将一百三十万张图片分类成为超过1000种类别。整个神经网络有超过6000万个参数和五十万个神经元,最终用softmax函数进行分类任务。达到了非常好的效果。
	
	在自然语言处理领域,Socher R.[3 Semi-Supervised Recursive Autoencoders for Predicting Sentiment Distributions]于2011年在EMNLP发表的文章使用半监督递归自编码器,以短语为单元,进行了情绪识别和分布预测,取得了比之前的技术更好的效果。
	
	Ngiam et al. [4] 用深度自编码器进行了语音和视频的多模态深度学习。他们分别利用了只有声音一个模态和声音和视频两个模态作为输入,利用RBM重建声音和视频两个模态的特征并寻找共同表达。最终结论为,用声音一个模态作为输入,而两个RBM分别重建声音和视频的特征寻找到的共同特征表达有最好的结果。
	
	 N Srivastava[5]利用多模态Deep Boltzmann Machines(DBM),作为模型,对图像和文字进行了多模态深度学习。获得了比深度自编码器和多模态DBN更好的结果。 利用多模态DBM学习得共同特征作为输入也在单模态学习中超过了原始的文字模态作为输入。
	 
	Xing et al. [6] 用dual-wing harmoniums的模型进行了文字和图像的多模态深度学习,他们使用了包含高斯隐含单元,高斯和泊松可见单元的线性RBM模型,最终结果明显好Gaussian-multinomial mixture 和 Gaussian-multinomial mixture LDA.
	
		本文的基本思路和灵感时来源于Ngiam[2011icmal Multimodal DL]于2011年在ICML发表的论文,在论文中他们提到,深度网络已经可以成功的被应用于单一模态的无监督特征学习,譬如文字,图像,声音等等。而他们工作的改进之处在于,提出了在多模态的情况下学习特征的方法,并且给出了具体的特征提取任务,说明了如何训练这个新的深度网络来解决这些所提出的任务。特别的事,他们还展示了跨模态(cross modality)特征学习,在这种情况下,在特征学习阶段充分利用多个模态的信息,我们就可以学习到更好的某个模态的特征。不仅如此,他们还展示了如何学习到多个模态的共同特征表达,并且利用具体例子对结果进行了评估,也就是分类器用单一声音模态来训练,而只用视频模态来进行测试,或者正好相反。 这种训练和测试利用不同的特征模态的方法就叫做学习多模态的共同特征表达的方法。 他们主要处理的是声音和图像两个模态的内容,利用人的语音发音和口型来识别人究竟说出了哪些词语,数据集主要包含了人对数字和英文字母的发音。
	 
	 
\section{工作介绍}
	本课题将着重研究如何将实验室获得的眼动仪数据和EEG信号相融合,使得利用两个模态作为输入信息的算法模型有相比于单纯利用EEG信号或者眼动仪数据的算法有更好的表现。目前国内外对语基于脑电信号的情绪研究和多模态学习分别都有深入的研究,但是将多模态学习利用在情绪识别,特别是在EEG信号的利用上还远远不足。
		
	我们通过对脑电信号和眼动仪信号的分析,对十一个被试人,每个人三组共三十三组的数据进行处理。结合多模态学习在其他领域的方法,提出了我们的多模态深度学习的算法模型,不但对比了提取到的在网络中不同深度的特征,还利用了不同的激励函数,差别符合预期,并且最终使得情绪识别的准确率有所提升。
\section{本章小结}
	本章主要介绍了基于多模态深度学习的情绪识别研究的目的和意义,以及多模态学习,脑电 情绪研究在国内外的研究现状。